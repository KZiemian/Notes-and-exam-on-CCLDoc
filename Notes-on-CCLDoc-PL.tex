% Autor: Kamil Ziemian

% --------------------------------------------------------------------
% Paczki
% --------------------------------------------------------------------
\RequirePackage[l2tabu, orthodox]{nag}  % Wykrywa przestarzałe i niedobre
% rozwiązania LaTeXa. Więcej jest w l2tabu engish version.
\documentclass[a4paper,11pt]{article}  % {rozmiar fontu}[klasa dokumentu]
\usepackage[utf8]{inputenc}  % Włączenie kodowania UTF-8, daje polskie znaki
\usepackage[polish]{babel}  % Nie pamiętam
\usepackage[MeX]{polski}  % Nie pamiętam
\usepackage{indentfirst}  % Wcięcie w pierwszym akapicie
\usepackage{microtype}  % Twierdzi, że poprawi rozmiar odstępów w tekście
\usepackage{vmargin}  % Pozwala na prostą kontrolę rozmiaru marginesów,
% za pomocą komend poniżej.
% ----------------------------
% MARGINS
% ----------------------------
\setmarginsrb
{ 0.7in} % left margin
{ 0.6in} % top margin
{ 0.7in} % right margin
{ 0.8in} % bottom margin
{  20pt} % head height
{0.25in} % head sep
{   9pt} % foot height
{ 0.3in} % foot sep
\usepackage{hyperref}  % Pozwala tworzyć hiperlinki i zamienia odwołania
% do bibliografii na hyperlinki.



% ----------------------------
% Paczki potrzebne do tego konkretnego dokumentu
% ----------------------------
\usepackage{listings}



% --------------------------------------------------------------------
% Dodatkowe ustawienia dla języka polskiego
% --------------------------------------------------------------------
\renewcommand{\thesection}{\arabic{section}.}
% Kropki po numerach rozdziału (polski zwyczaj topograficzny)
\renewcommand{\thesubsection}{\thesection\arabic{\subsection}}
% Brak kropki po numerach podrozdziału



% --------------------------------------------------------------------
% Tytuł, autor, data
% --------------------------------------------------------------------
\title{Notatki o~używaniu CCLDOC}




% ####################################################################
% Początek dokumentu
\begin{document}
% ####################################################################


% ##########
% Ustawienie trybu formatowania kodu do konkretnej rodziny języków.
\lstset{language=Lisp}
% ##########


\maketitle % Tytuł, autor, data

\tableofcontents



% ##############################
\section{Źródła i~środowiska pracy}


% ####################
\subsection{Źródła}

Materiałów do~CCLDoca nie jest wiele, więc korzystałem z~tych najbardziej podstawowych. Konto GitHubowe jest głównym źródłem kodu z~którego korzystam


% ####################
\subsection{Środowisko pracy}

Kod testowałem w Clojure Common Lisp (od teraz w~skrócie CCL)
na~systemie Ubuntu 16.04. Steel Bank Common Lisp\footnote{Którego
  zazwyczaj używam.} (dalej SBCL) nie jest w~pełni kompatybilny
w~obecnej chwili z~CCLDoc, miejmy nadzieję, że~pod koniec prac nad
tymi notatkami powstanie kod czyniący CCLDoc kompatybilnym z~tą
implementacją Common Lispa.

Kod pisałem i~edytowałem w~GNU Emacsie~24 i~25. NA koncie GitHuba
\href{https://github.com/Clozure/ccldoc}{Clozure}\footnote{Pełny adres
  to \href{https://github.com/Clozure/ccldoc}
  {https://github.com/Clozure/ccldoc}, prościej jednak wygooglować.
  Tam również znajduje~się plik \emph{.gitignore}, który ma czynić
  SBCL kompatybilnym z~CCLDoc.} dostępny jest plik \verb+ccldoc.el+,
który wprowadza tryb Emacsa do~edycji plików CCLDoc. Oto jak
skonfigurowałem Emacsa by~z~niego korzystał. W~swoim katalogu
\verb+~/.emacs.d/+ mam katalog \verb+elisp+ i~w~nim umieściłem plik
\verb+ccldoc.el+, następnie dodałem do~swojego pliku \verb+init.el+
kod
\begin{lstlisting}
;; Tryb dla CCLDOC
(load "~/.emacs.d/elisp/ccldoc.el")
\end{lstlisting}
Przy takiej konfiguracji, aby~Emacs od razu po wejściu do pliku CCLDoc, np.~\verb+przyklad-s01-01.ccldoc+, musimy w~pierwszej linii pliku umieścić
\begin{lstlisting}
;;;   -*- Lisp -*-
\end{lstlisting}
Taki kształt

Każdą
\begin{lstlisting}
(load "home:quicklisp;setup")
(ql:quickload :ccldoc)
\end{lstlisting}

\begin{lstlisting}
There is no package named "QL" .
[Condition of type CCL::NO-SUCH-PACKAGE]
\end{lstlisting}









% ####################################################################
% Koniec dokumentu
\end{document}
% ####################################################################
